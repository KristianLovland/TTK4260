\subsection{PLS}
PCR har en åpenbar svakhet: Metoden baserer seg på antagelsen om at de prinsipale komponentene, kun utledet fra $x$, også forklarer variasjon i $y$. PLS ligner på PCR, men kvitter seg med denne antagelsen, og sikter heller på å finne komponenter som forklarer variasjon i $y$. Metoden beskrives ikke av noen enkel likning, men baserer seg på å regne ut latente strukturer (PLS forklares gjerne som Projection into Latent Structures) som gjør at dataen vår kan skrives på formen

\begin{align}
	X & = T P^T + E \\
	Y & = U Q^T + F
\end{align}

Her har $P$ og $T$ samme tolkning som i PCA anvendt på $X$ (ladning og score), og $U$ og $Q$ forstås på tilsvarende vis i rommet $Y$ lever i. $E$ og $F$ er feil, som antas uavhengige med identisk fordeling. Akkurat hva retningene kolonnene i $T$ og $U$ beskriver er litt vanskeligere å beskrive enn for PCA, men overordnet kan man tenke på det som at de ligner score-matrisene utregnet i PCA, men at man her også sørger for at $T$ og $U$ har stor kovarians ($T$ forklarer $U$.

	For en dypere forståelse av betydningen kan det være nyttig å undersøke NIPALS-algoritmen for PLS, beskrevet under. Denne har mange likhetstrekk med den enklere NIPALS-algoritmen som finner PCA, men regner nødvendigvis ut flere matriser (alt gjøres parallelt). Som tidligere antar man at $X$ og $Y$ er sentrert.

\begin{algorithm}
	\caption{NIPALS for PLS}\label{alg:nipals_pls} \begin{algorithmic}[1] \Procedure{NIPALS}{$X, Y, n_\textrm{PLS}, t_\textrm{tol}$}\Comment{Numerisk utregning av PLS}
	\State $i \gets 1$
	\While{$i < n_\textrm{PLS}$}
	\State Initialisér $u_i$ (tilfeldig, eller som en av kolonnene i $Y$)
	\While{Endring i $t_i > t_\textrm{tol}$}
	\State $w_i \gets \frac{X^T u_i}{|| X^T u_i ||_2}$ \Comment{Projeksjon av $X$ på $u_i$}
	\State $t_i \gets X w_i$ \Comment{Projeksjon av $X$ på $w_i$}
	\State $q_i \gets \frac{Y^T t_i}{|| Y^T t_i ||_2}$ \Comment{Projeksjon av $Y$ på $t_i$}
	\State $u_i \gets Y q_i$ \Comment{Projeksjon av $Y$ på $q_i$}
	\EndWhile
	\State $p_f \gets \frac{X^T t_f}{t_f^T t_f}$
	\State $b_f \gets \frac{u_f^T t_f}{t_f^T t_f}$
	\State $X \gets X - t_i p_i^T$ \Comment {Deflasjon av $X$}
	\State $Y \gets Y - b_i^T q_i^T$ \Comment {Deflasjon av $Y$}
	\State $i \gets i + 1$
	\EndWhile
\EndProcedure
\end{algorithmic}
\end{algorithm}
Det er mye som kan sies om matrisene som returneres, men en av de viktigste plottene å undersøke er plottet av $t$ vs $u$. Dette representerer den lineære regresjonen mellom de to nye koordinatsystemene PLS definerer for $X$ og $Y$, og forteller noe om hvordan komponentene i $X$ forklarer komponentene i $Y$.


