\subsection{Bias vs. varians}
Dette er en avveining man ikke slipper unna når men bedriver estimering. La oss bruke MSE for å illustrere dette. La 

\begin{equation}
\begin{array}{l}{\mathcal{V}:=\widehat{\theta}-\mathbb{E}[\widehat{\theta}]} \\ {\mathcal{B}:=\mathbb{E}[\widehat{\theta}]-\theta}\end{array}
\end{equation}

\begin{align}
\mathbb{E}\left[\|\hat{\theta}-\theta\|^{2}\right] &=\mathbb{E}\left[\|\widehat{\theta}-\mathbb{E}[\widehat{\theta}]+\mathbb{E}[\widehat{\theta}]-\theta\|^{2}\right] \\ \nonumber
&=\mathbb{E}\left[\|\mathcal{V}+\mathcal{B}\|^{2}\right]\\ \nonumber
&=\mathbb{E}\left[(\mathcal{V}+\mathcal{B})^{T}(\mathcal{V}+\mathcal{B})\right] \\ \nonumber
&=\mathbb{E}\left[\|\mathcal{V}\|^{2}+\|\mathcal{B}\|^{2}+2 \mathcal{V}^{T} \mathcal{B}\right] \\ \nonumber
&=\mathbb{E}\left[\|\mathcal{V}\|^{2}\right]+\|\mathcal{B}\|^{2}
\end{align}

Denne avveiningen henger sammen med hvor komplisert man gjør forklaringsmodellen $f(u_t; \theta)$. Om man gjør den veldig komplisert vil man kunne følge dataen nøyaktig, men man vil være utsatt for at dette ikke lar seg generalisere til andre datasett \textbf{overfitting}. Dette svarer til lav bias, men stor varians. Om modellen er for enkel vil man få en enkel modell som generaliserer, men man vil også kunne unngå å beskrive viktig struktur i dataen. Dette er \textbf{underfitting}, og svarer til liten varians, men stor bias.

Det finnes flere metoder som forsøker å gjøre denne avveiningen. Noen av dem er
\begin{itemize}
\item Akaikes informasjonskriterium
\item Det Bayesiske informasjonskriteriumet
\item Minimum lengde-beskrivelse
\end{itemize}
